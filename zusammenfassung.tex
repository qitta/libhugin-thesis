\documentclass{scrartcl}
\thispagestyle{empty}
\usepackage{geometry}
\geometry{top=20mm, left=30mm, right=30mm, bottom=20mm}
\usepackage[utf8]{inputenc}
\usepackage[ngerman]{babel}
\usepackage[onehalfspacing]{setspace}
\RequirePackage{garamondx}
\begin{document}
\begin{center}
    \LARGE \textbf{Zusammenfassung} \\
    \vspace{.1in}
    \large der Bachelorarbeit \\
    \vspace{.1in}
    \small ,,Algorithmik und Evaluation des Filmmetadaten Such- und
    Analysesystems libhugin" \\
    \vspace{.1in}
    \small \textit{Christoph Piechula, \today}
    \vspace{.2in}
\end{center}

\textit{Libhugin} ist eine in der Projektarbeit \cite{cpiechula} entwickelte
modulare Bibliothek für die Suche und Analyse von Filmmetadaten. Die Bibliothek
verfolgt im Vergleich zu den bestehenden Filmmetadaten Such- und
Analysewerkzeugen einen modularen Ansatz. Filmmetadaten gibt es in
verschiedenen Sprachen und Ausprägungen. Durch den modularen Ansatz wird dem
Benutzer die Möglichkeit gegeben, das System durch den Entwurf von Plugins
\footnote{Eine Modul welches nicht direkter Bestandteil des Softwarekerns ist}
besser an die eigenen Bedürfnisse anzupassen. Hier gibt es verschiedene
Pluginarten, welche neben der Metadatenbeschaffung auch für das direkte
Verarbeiten oder Exportieren der Metadaten zuständig sind. Neben der
Metadatenbeschaffung gibt es auch die Möglichkeit einer Metadatenaufbereitung.
Hierdurch lassen sich Metadaten auch nachträglich analysieren und Änderungen an
diesen vornehmen. Dieser Teil der Bibliothek ist ebenso durch den Entwurf eigener
Plugins erweiterbar. Zusätzlich zur Bibliothek wurden Werkzeuge zu
Demonstrationszwecken und Testen der Bibliothek entwickelt.
\\

Diese Arbeit thematisiert die hinter \textit{libhugin} stehenden Ansätze und
Algorithmen und untersucht anhand von Stichproben die bisher, auch während der
Entwicklung, getroffenen Annahmen zu den Metadaten und Metadatenquellen.
Einleitend wird ein Überblick über die Software-Architektur der Bibliothek
vermittelt.  Anschließend werden die technischen Grundüberlegungen erläutert.
Der Hauptteil befasst sich mit der Algorithmik und der Analyse der Bibliothek, 
sowie der Analyse der Metadaten und Metadatenquellen. Hier wird auf die
Entwicklung der Standard--Algorithmen zur Filmsuche eingegangen. Es wird auch
explizit auf die Algorithmik von Features wie beispielsweise der ,,IMDb-ID
Suche`` oder der ,,Unschärfesuche`` eingegangen. Anschließend werden bestimmte
Kriterien wie das Antwortverhalten der implementierten Metadatenquellen-Plugins
oder auch die Verarbeitungszeiten der Metadaten untersucht. Gegen Ende der
Arbeit werden die von den Metadatenquellen erhobenen Daten anhand einer
Stichprobe untersucht, um die bisherigen Annahmen zu bestätigen oder zu
revidieren. Die Analyse zeigt beispielsweise die Unterschiede verschiedener
Metadatenquellen auf, aber auch wie die Vollständigkeit der Metadaten
je nach Onlinequellen variiert.  
\\

Abschließend werden anhand der gewonnenen Daten noch bestehende Probleme
identifiziert sowie Ideen und Verbesserungsvorschläge für die weitere
Entwicklung genannt.
\\
\\
\vspace*{-2em}\large\textbf{Literatur}
\renewcommand*{\refname}{} 
\vspace*{-1em}
\small
\bibliographystyle{unsrt}
\bibliography{refs.bib}
\end{document}
